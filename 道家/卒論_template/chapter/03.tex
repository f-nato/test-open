\chapter{X線分布再構成原理}
システムにピンホールを通じてある視線から入射する軟X線は,プラズマ中に生じた高速電子分布をその視線において線積分した値と捉えることができ,ここに分布の軸対称性を仮定することで,二次元逆変換問題としての計算機処理が可能となる。再構成対象の領域$\mathrm{D}$を$\mathrm{K}$個の格子に分割し,これを$\mathrm{M}$本の視線で観測することを考える。電子分布$\mathrm{E}$は各ピクセルの発光強度を要素$\mathrm{E_k}$に持つ$\mathrm{K}$行$1$列のベクトルであり,$\mathrm{S_i}$が$i$番目の視線における発光強度の積分値を表す$\mathrm{M}$次ベクトル$S$は,$\mathrm{L_{ij}}$が$j$番目の視線における$i$番目のピクセルの重みを表す行列$\mathrm{L}$を用いて以下の通り記述できる。
\begin{align}
	S &= LE\\
	L^{T}S &= L^{T}LE 
	\label{eq;s=le}
\end{align}
ここで$\mathrm{S, L}$は既知であるが,多くの二次元測定で観測センサの個数$M$は求めたい変数の個数$K$より少なく$M < K$となりがちで,$L^{T}L$は逆行列をもたない非正則行列となり方程式は解不定である。そこで無数の解から再構成像を選択するために,二乗誤差平均$\epsilon^2 = \mathrm{|S - LE'|^2/M}$をはじめ様々な制約を$E'$に課して決定する。このような手法は正則化と呼ばれ,大別してモデル関数群の線形結合を仮定して変数を削減し係数決定問題に持ち込む級数展開法と,ピクセル値をそのまま変数として扱うペナルティ関数法がある。
\section{正則化法}
\subsection{Tikhonov-Phillips正則化}
二乗誤差$\epsilon^2$が等しい解集合の中から,真の分布がなめらかであるという仮定のもとそのような画像を再構成像として選択するのがペナルティ関数法である。そのために,画像の粗さを評価する関数$P(f)$を用いて以下の$\Lambda$を最小化することを考える。
\begin{equation}
	\Lambda = \gamma P(f) + \epsilon^2
	\label{eq;lambda}
\end{equation}
ここで$P(f) = \int_{D}|\nabla^nf(r)|^2dr$とするのがTikhonov-Phillips正則化であり,$K$次ラプラシアン行列を$C$として$P(f) = |Cf|^2$と近似すれば,最小の$\Lambda$を与える$f$はラグランジュの未定乗数法より次の式で表される。
\begin{equation}
	f = (L^{T}L + M\gamma C^{T}C)^{-1}S
\end{equation}
\subsection{最小Fisher情報量法}
ペナルティ関数として
\begin{equation}
	P(f) = \int_{D}\frac{|\nabla^nf(r)|^2}{f(r)}dr
\end{equation}
を選択する場合\cite{fisher},強度$f$の小さい部分において$P$の分子が減少,ペナルティが増大することで正則化が強まり,ピークから離れた箇所での高周波成分が丸められる。前述の線形関数が画像全体に一様に正則化を求めるのに対して,この手法では強度の小さい部分により強い正則化をかけることになり,主にオフピーク領域での画像の改善が期待される。この離散表現は,$K$次対角行列$W$を用いて
\begin{align}
	f_{n+1} &= (L^{T}L + M\gamma C^{T}W_{n}C)^{-1}S\\
	W_{n, ij} &= \delta_{ij} \cdot \frac{1}{f_{i}}
\end{align}
とし$W_n$が収束するまで反復計算することで実現できる。
\subsection{最小GCV基準}
式(\ref{eq;lambda})に示す$\gamma$は正則化の強度を定めるパラメータであり,大きければ大きいほど正則化項$\gamma P$が増大し画像はよりなめらかになる一方で,相対的に誤差項$\epsilon^2$が小さくなり解が現実と乖離する可能性が高まる。最小GCV基準では,次の量GCVが最小となる$\gamma$を選択する。
\begin{equation}
	GCV(\gamma) = \frac{\epsilon^2}{1 - \frac{1}{M}\Sigma \frac{\sigma_i^2}{\sigma_i^2 + \gamma M}}
\end{equation}
ただし$\sigma_i$は$LC^{-1}$の特異値である。定性的には,$\gamma$が大きい領域では分子の残差$\epsilon^2$が大きくなりGCVは上昇する一方で,$\gamma$が小さい領域でも分母が大きくなるためGCVが上昇する。この間の点を最適な$\gamma$と定めている。
\section{フィルタ処理法}
\subsection{線形平均化フィルタと非線形平均化フィルタ}
ペナルティ関数法は二乗誤差がある値に等しい中での最尤推定であるが,式(\ref{eq;lambda})の右辺第二項を直接抑えるための前処理として統計的雑音処理を加え,画像中の小さなピークがより信頼性の高い状態で再構成されることを期待する。白色性とガウス性を仮定した雑音を低減するために,ある画素の周囲に重みを与えその結合によって画素の値を更新するのが平均化フィルタであるが,線形平均化フィルタは対象と周囲の計9画素について等しく重みを与えるために,画像の鮮鋭度は十分に保存されない。一方で,図\ref{fig:nlmfilt}に示すように,対象画素を含むブロックと周囲のプロックの類似度からより類似度の高いものに大きな重みを与える非線形平均化フィルタ(NLM)は,画像のエッジをより保つことが可能である。更新対象画素$p$に対する比較対象画素$q$の類似度$w(p, q)$は,それぞれを中心とする同じサイズのブロック間の二乗ノルム$||v(p) - v(q)||$に対して指数関数的に減衰する次式の形で決定される。\cite{nlmfilter}

\begin{align}
	p &= \Sigma_{q\in S} w(p,q) * q\\
	w(p,q) &= \frac{1}{Z(p)} \exp\left(-\frac{\max(||v(p) - v(q)||^2 - 2\sigma^2, 0)}{\sigma^2}\right)\\
	Z(p) &= \Sigma_{q\in S} \exp\left(-\frac{\max(||v(p) - v(q)||^2 - 2\sigma^2, 0)}{\sigma^2}\right)
\end{align}

\begin{figure}[H]
	 \centering
	 \includegraphics[scale=0.6]{fig/nlmfilt.png}
	 \caption{NLMの処理:画素間の類似度$w(p,q)$は,それぞれを囲む領域をベクトル化した$v(p), v(q)$のノルムから決定される。}
	 \label{fig:nlmfilt}
\end{figure}