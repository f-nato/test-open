\chapter{結論}
プラズマ中の高エネルギー電子分布の中でも特にあまり検出できないX点加熱などの電子局所加熱を空間的・時間的により解像度の高い状態で検出するために,MCPを用いた軟X線カメラシステムと再構成ソフトウェアの開発を行った。
その結果,
\begin{itemize}
  \item [(1)] 最小Fisher情報量を用いた非線形再構成により,画像中オフピーク領域の再構成誤差が73\%にまで減少することを発見した
  \item [(2)] NLMフィルタを前処理として再構成を行うことで,ピーク領域の再構成誤差が80\%にまで減少することを発見した
  \item [(3)] NLMフィルタが低減するノイズの周波数はX点加熱のもつ特徴的な周波数と分離されていることを発見した
\end{itemize}
といった開発面での成果と,
\begin{itemize}
  \item [(4)] X点加熱を示唆する発光を世界で初めて撮影できた
  \item [(5)] ダウンストリーム加熱が磁力線に沿ってプラズマ全体に広がる過程を実証した
\end{itemize}
という物理面での成果を得た。
今回開発したシステムはカメラを増設し視点数を最大12まで増やすことが可能であり,幅広いエネルギー帯の電子を同時に観測すればプラズマの加熱機構がより詳細に判明するであろう。あるいは軸対称性を仮定しない三次元再構成や,医用CTで実用化されるようなノイズにより強い二次元再構成を目指すこともできるはずである。
本研究以外のノイズリダクション技術としては深層学習によるガウスノイズ除去手法が検討されても良い。今後もより信頼性の高い計測手法を目指し,多角的にプラズマの加熱機構の解明に務める。